\documentclass[a4paper,12pt]{article} % Establece el tamaño del papel y la fuente

\usepackage[utf8]{inputenc} % Codificación de entrada
\usepackage{amsmath} % Paquete para matemáticas
\usepackage{graphicx} % Paquete para incluir gráficos
\usepackage{geometry} % Paquete para ajustar márgenes
\geometry{left=1in, right=1in, top=1in, bottom=1in} % Ajusta los márgenes
\usepackage{lipsum} % Paquete para texto de ejemplo

\title{Los Números Primos}
\author{Sr. Antonio de la Cruz}
\date{\today}

\begin{document}

\maketitle

\begin{abstract}
    En este documento, exploraremos el concepto de números primos, su importancia en matemáticas y algunos ejemplos relevantes.
\end{abstract}

\tableofcontents % Incluye una tabla de contenido

\section{Introducción}
Los números primos son números naturales mayores que 1 que solo tienen dos divisores: 1 y el propio número. Son fundamentales en la teoría de números y tienen aplicaciones en criptografía, teoría de números y matemáticas puras.

\section{Definición de Números Primos}
Un número primo es un número entero positivo mayor que 1 que no puede ser dividido exactamente por ningún otro número excepto 1 y él mismo. Por ejemplo:

\begin{itemize}
    \item 2 es un número primo porque solo es divisible por 1 y 2.
    \item 3 es un número primo porque solo es divisible por 1 y 3.
    \item 4 no es un número primo porque es divisible por 1, 2 y 4.
\end{itemize}

\section{Propiedades de los Números Primos}
Los números primos tienen varias propiedades interesantes:

\begin{enumerate}
    \item Todos los números primos mayores que 2 son impares.
    \item El número 2 es el único número primo par.
    \item Existen infinitos números primos, como lo demostró el matemático griego Euclides.
\end{enumerate}

\section{Ejemplos de Cálculos}
Aquí hay un ejemplo de cómo encontrar números primos usando la Criba de Eratóstenes:

\begin{quote}
    \textbf{Criba de Eratóstenes:} Para encontrar todos los números primos hasta un número \( n \), se elimina múltiplos de cada número primo comenzando desde 2.
\end{quote}

\begin{equation}
    \text{Por ejemplo, para encontrar primos menores que 30:}
\end{equation}

\begin{align*}
    \text{1. Empieza con la lista } & [2, 3, 4, 5, 6, 7, 8, 9, 10, 11, 12, 13, 14, 15, 16, 17, 18, 19, 20, 21, 22, 23, 24, 25, 26, 27, 28, 29, 30] \\
    \text{2. Elimina múltiplos de 2:} & [2, 3, 5, 7, 9, 11, 13, 15, 17, 19, 21, 23, 25, 27, 29] \\
    \text{3. Elimina múltiplos de 3:} & [2, 3, 5, 7, 11, 13, 17, 19, 23, 25, 29] \\
    \text{4. Los números restantes son primos.}
\end{align*}

\section{Conclusión}
Los números primos son un área de estudio fascinante en matemáticas con muchas aplicaciones prácticas y teóricas. Comprender sus propiedades y cómo encontrarlos es fundamental para diversos campos.

\section{Referencias}
Para más información sobre números primos, consulta los siguientes recursos:

\begin{itemize}
    \item \textit{Introducción a la teoría de números} de David M. Bressoud.
    \item \textit{La Criba de Eratóstenes} en Wolfram MathWorld.
\end{itemize}

\end{document}
